\documentclass{article}
\usepackage[paper=a4paper,left=3cm, right=3cm, top=2cm]{geometry}
\usepackage[utf8]{inputenc}

\title{\textbf{Nanoparticles : Medical Imaging Advancement\\Update 8}}
\author{Sajal Harmukh \\ \textit{Department of Biomedical Engineering} \\ National Institute of Technology, Raipur }
\date{11\textsuperscript{th} March 2022}

\begin{document}

\maketitle

\section*{Aim}
\large
To study and summarise the recent advancement and the emerging research on the use of nanoparticles in the biomedical imaging and to lookout for an early detection and diagnosis of the disease, which is the driving force in the field of medical imaging.

\section*{Introduction}
Nanomaterials, such as nanoparticles, nanorods, nanosphere, nanoshells, and nanostars, are
very commonly used in biomedical imaging.They make excellent drug carriers, imaging contrast agents, photothermal agents, photoacoustic agents, and radiation dose enhancers, among other applications. Recent advances in nanotechnology have led to the use of nanomaterials
in many areas of functional imaging, cancer therapy.\\ Nanoparticles 1 – 100 nm in diameter have dimensions comparable to biological functional units. Diverse surface chemistries, unique magnetic properties, tunable absorption and emission properties, and recent advances in the synthesis and engineering of various nanoparticles suggest their potential as probes for early detection of diseases such as cancer.
 


\section*{Week 8 }
\subsection*{ Nanoparticles in PET/SPECT imaging applications}
Positron emission tomography (PET) is a powerful and widely used nuclear medicine technology with high tissue penetration, high sensitivity and real-time quantitative imaging analysis. Besides anatomic information, PET may also provide biological information at the molecular level based on nuclide tracking. Single photon emission computed tomography (SPECT) is another widely used nuclear medicine technology with similar advantages as PET imaging, they can detect abnormal biochemical function before changes in anatomy. Limitations of PET/SPECT include high cost and radioactive exposure.\\ Nuclides with a reasonably long half-life are needed for nanoparticle PET/SPECT imaging tracers. The commonly used radioisotope for PET in clinical practice is Fluorine-18 with a half-life of 109.8 minutes, which is generally too short to apply in nanoparticles due to the time needed for preparation and cellular uptake.196 Thus, other nuclides with longer half-lives are needed to prepare various nanoparticles. Examples include copper-64 with a half-life of 12.7 hours, Indium-111 with a half-life of 2.8 days and Iodine-124 with a half-life of 4.2 days.197–199 With the exception of incorporation of a radionuclide tracer, nanoparticles for PET imaging are similar in structure to other described nanoparticles used for medical imaging. Compared with PET, radionuclides in SPECT generally have longer half-lives. The common radionuclide for SPECT in clinical use is Technetium-99m with a half-life of 6 hours. Nanoparticle structures for SPECT imaging are similar to structures for PET imaging. For example, Technetium-99m labeled macro-aggregated albumin particles are commonly used to quantify tumor volumes and pulmonary shunt fractions in the liver.200 Other radionuclides used in nanoparticles 201 include Indium-111, Iodine-123, 125 and 131, and Gold-198, 199.
\begin{itemize}
     
    
\end{itemize}





 
 \end{document}
