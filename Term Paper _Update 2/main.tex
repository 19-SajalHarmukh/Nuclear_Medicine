\documentclass{article}
\usepackage[paper=a4paper,left=3cm, right=3cm, top=2cm]{geometry}
\usepackage[utf8]{inputenc}

\title{\textbf{Nanoparticles : Medical Imaging Advancement\\Update 2}}
\author{Sajal Harmukh \\ \textit{Department of Biomedical Engineering} \\ National Institute of Technology, Raipur }
\date{28 January 2022}

\begin{document}

\maketitle

\section*{Aim}
\large
To study and summarise the recent advancement and the emerging research on the use of nanoparticles in the biomedical imaging and to lookout for an early detection and diagnosis of the disease, which is the driving force in the field of medical imaging.

\section*{Introduction}
Nanomaterials, such as nanoparticles, nanorods, nanosphere, nanoshells, and nanostars, are
very commonly used in biomedical imaging.They make excellent drug carriers, imaging contrast agents, photothermal agents, photoacoustic agents, and radiation dose enhancers, among other applications. Recent advances in nanotechnology have led to the use of nanomaterials
in many areas of functional imaging, cancer therapy.\\ Nanoparticles 1 – 100 nm in diameter have dimensions comparable to biological functional units. Diverse surface chemistries, unique magnetic properties, tunable absorption and emission properties, and recent advances in the synthesis and engineering of various nanoparticles suggest their potential as probes for early detection of diseases such as cancer.
 
\section*{Week 2}
Medical imaging technology often plays the most important role in the early detection and therapeutic response assessment of various diseases. Imaging modalities in current use include X-ray radiography, computed tomography (CT), magnetic resonance imaging (MRI), ultrasound (US), positron emission tomography (PET), single photon emission computed tomography (SPECT), and fluorescence imaging.\\ For instance, the two-year endurance pace of gastrointestinal malignant growth patients for the individuals who profited from early location has been seen to be a lot higher than in those without early identification (92.3percentage VS 33.3 percentage). Moreover, the ten-year death rate for bosom disease patients who profited from early discovery was decreased by 17-28 percentage.\\
To further develop sore discovery, regularly more than one imaging methodology is joined. To get more exact anatomic and utilitarian data, clinical imaging contrast specialists are utilized to recognize ordinary tissue and strange injuries. For customary clinical imaging contrast specialists, cancer discovery is restricted by the spatial goal created by the imaging equipment, for example, the capacity of difference improved CT to distinguish a hyper-vascular hepa-toma as little as 3 mm. Right now utilized clinical imaging contrast specialists are generally little particles that show quick digestion, and have vague appropriation and expected unwanted \\poison levels.
\\Nanomaterials have unique optical, electrical and/or magnetic properties at the nanoscale, and these can be used in the fields of electronics and medicine, amongst other scenarios. Nanomaterials are unique as they provide a large surface area to volume ratio. Unlike other large-scaled engineered objects and systems, nanomaterials are governed by the laws of quantum mechanics instead of the classical laws of physics and chemistry. In short, nanotechnology is the engineering of useful objects and functional systems at the molecular or atomic scale .
\\Inside the biomedical local area, somewhat bigger particles are regularly characterized as nanoparticles also, inferable from a comparability in size to significant normally happening nanoparticles, for example, infections. At these aspects, nanoparticles show extraordinary properties that might be unmistakable from the two atoms and mass solids. Without the guide of exogenous focusing on ligands, nanoparticles have been seen to target growths inactively through the upgraded porousness and maintenance impact (EPR) , or explicit tissues, for example, the lymphatic framework through sub-atomic sieving


 
  

\end{document}
