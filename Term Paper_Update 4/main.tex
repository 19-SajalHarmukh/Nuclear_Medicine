\documentclass{article}
\usepackage[paper=a4paper,left=3cm, right=3cm, top=2cm]{geometry}
\usepackage[utf8]{inputenc}

\title{\textbf{Nanoparticles : Medical Imaging Advancement\\Update 4}}
\author{Sajal Harmukh \\ \textit{Department of Biomedical Engineering} \\ National Institute of Technology, Raipur }
\date{11\textsuperscript{th} February 2022}

\begin{document}

\maketitle

\section*{Aim}
\large
To study and summarise the recent advancement and the emerging research on the use of nanoparticles in the biomedical imaging and to lookout for an early detection and diagnosis of the disease, which is the driving force in the field of medical imaging.

\section*{Introduction}
Nanomaterials, such as nanoparticles, nanorods, nanosphere, nanoshells, and nanostars, are
very commonly used in biomedical imaging.They make excellent drug carriers, imaging contrast agents, photothermal agents, photoacoustic agents, and radiation dose enhancers, among other applications. Recent advances in nanotechnology have led to the use of nanomaterials
in many areas of functional imaging, cancer therapy.\\ Nanoparticles 1 – 100 nm in diameter have dimensions comparable to biological functional units. Diverse surface chemistries, unique magnetic properties, tunable absorption and emission properties, and recent advances in the synthesis and engineering of various nanoparticles suggest their potential as probes for early detection of diseases such as cancer.
 


\section*{Week 4 }
\subsection*{ Synthesis of nanoparticles for molecular imaging}
\begin{itemize}
    \item Gold nanoparticles - |\\norganic nanoparticles are arising as adaptable devices in imaging because of their one of a kind compound, physical and optical properties. Gold nanoparticles were found > 100 years prior and, attributable to their surface science, projected biocompatibility, somewhat low momentary poisonousness, high nuclear number and high X-beam retention coefficient, gold nanoparticles have gotten critical interest as of late for use in different imaging advancements. There are powerful and easy manufactured strategies for creating gold nanoparticles with exact command over the molecule size and shape.\\ 
    \item Amalgamation - Round gold nanoparticles with sizes going from a couple to a few hundred nanometers can be incorporated advantageously in watery or natural solvents with a serious level of accuracy and precision . By and large, decrease of gold salts (e.g., AuCl4-) prompts the nucleation of gold particles As gold nanoparticles are not steady, a balancing out specialist is required that is genuinely adsorbed or synthetically bound to the gold surface. The age of gold nanoparticles in fluid medium commonly utilizes either trisodium citrate or sodium borohydride as lessening specialists. Definitively controlling citrate fixation brings about the development of little and uniform gold nanoparticles.
\end{itemize}
X-ray computer tomography (CT), is a commonly used diagnostic imaging tool offering broad availability and relatively modest cost. X-ray CT is used to visualize tissue density differences that provide image contrast by X-ray attenuation between soft tissues and electron-dense bone. It is desirable to enhance the contrast of diseased tissue with the use of X-ray contrast agents to increase the contrast between normal and cancerous tissue





 
 \end{document}
