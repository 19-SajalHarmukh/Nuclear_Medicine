\documentclass{article}
\usepackage[paper=a4paper,left=3cm, right=3cm, top=2cm]{geometry}
\usepackage[utf8]{inputenc}

\title{\textbf{Nanoparticles : Medical Imaging Advancement\\Update 5}}
\author{Sajal Harmukh \\ \textit{Department of Biomedical Engineering} \\ National Institute of Technology, Raipur }
\date{18\textsuperscript{th} February 2022}

\begin{document}

\maketitle

\section*{Aim}
\large
To study and summarise the recent advancement and the emerging research on the use of nanoparticles in the biomedical imaging and to lookout for an early detection and diagnosis of the disease, which is the driving force in the field of medical imaging.

\section*{Introduction}
Nanomaterials, such as nanoparticles, nanorods, nanosphere, nanoshells, and nanostars, are
very commonly used in biomedical imaging.They make excellent drug carriers, imaging contrast agents, photothermal agents, photoacoustic agents, and radiation dose enhancers, among other applications. Recent advances in nanotechnology have led to the use of nanomaterials
in many areas of functional imaging, cancer therapy.\\ Nanoparticles 1 – 100 nm in diameter have dimensions comparable to biological functional units. Diverse surface chemistries, unique magnetic properties, tunable absorption and emission properties, and recent advances in the synthesis and engineering of various nanoparticles suggest their potential as probes for early detection of diseases such as cancer.
 


\section*{Week 5 }
\subsection*{ Quantum dots}
Quantum dabs (QDs) are fluorescent semiconductor nanocrystals (∼ 1 - 100 nm) with extraordinary optical and electrical properties . Contrasted and natural colors and fluorescent proteins, QDs have close solidarity quantum yields and a lot more noteworthy splendor than most colors (10 - multiple times). Quantum dabs likewise show wide assimilation attributes, a restricted linewidth in emanation spectra, nonstop and tunable outflow maxima because of quantum size impacts, a generally lengthy fluorescence lifetime (5 to > 100 ns contrasted and 1 - 5 ns in natural colors) and insignificant photobleaching (100 - multiple times not exactly fluorescent colors) over minutes to hours .
\begin{itemize}
    \item Synthesis - uantum dots have a synthetic history of ∼ 17 years, with synthesis described by Ekimov and Onuschenko and Efros and Efros in 1982 . Extensive effort has been invested ever since to enlarge the spectrum, functionality and biocompatibility of QDs. Quantum dots, like any other nanoparticles, have a large number of atoms on the surface containing vacant atomic or molecular orbitals. Bawendi and co-workers developed a synthetic method for the synthesis of QDs containing cadmium sulfide (CdS), cadmium selenide (CdSe), or cadmium telluride (CdTe) . Quantum dots have been developed with a metalloid crystalline core (e.g., CdSe) and a shell (e.g., ZnS) that shields the core. It was assumed that the capping of QDs with wider band gap semiconducting materials such as zinc sulfide (ZnS) would improve the photoluminescence efficiency.
    \\Utilization of ZnS covering additionally diminishes the oxidative photobleaching of QDs  and enormously upgrades the surface restricting properties of CdS, CdSe and CdTe-center QDs to ligands, for example, phosphines and amines, in this way further developing colloidal soundness . Despite the fact that strategies for the amalgamation of QDs in fluid medium have been created, these techniques have seldom yielded colloidal strength with high monodispersity contrasted and QDs produced within the sight of planning hydrophobic ligands. In a run of the mill shell development technique for the combination of QDs, a selenium forerunner (e.g., trioctylphosphine (TOP)- selenide) is added quickly to a hot arrangement of a cadmium antecedent (e.g., cadmium oleate) and a planning ligand (e.g., hexadecylamine) under latent environmental circumstances followed by the option of an answer of diethyl zinc. Hexamethyldisilathiane in TOP is then added gradually to work on the photoluminescence proficiency. Cleaning utilizing fluid extraction or precipitation prompts the recuperation of unadulterated (CdSe)ZnS nanocrystal. The choice of QD center structure is directed by the ideal frequency of emanation.
    
\end{itemize}





 
 \end{document}
