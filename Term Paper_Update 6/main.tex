\documentclass{article}
\usepackage[paper=a4paper,left=3cm, right=3cm, top=2cm]{geometry}
\usepackage[utf8]{inputenc}

\title{\textbf{Nanoparticles : Medical Imaging Advancement\\Update 6}}
\author{Sajal Harmukh \\ \textit{Department of Biomedical Engineering} \\ National Institute of Technology, Raipur }
\date{25\textsuperscript{th} February 2022}

\begin{document}

\maketitle

\section*{Aim}
\large
To study and summarise the recent advancement and the emerging research on the use of nanoparticles in the biomedical imaging and to lookout for an early detection and diagnosis of the disease, which is the driving force in the field of medical imaging.

\section*{Introduction}
Nanomaterials, such as nanoparticles, nanorods, nanosphere, nanoshells, and nanostars, are
very commonly used in biomedical imaging.They make excellent drug carriers, imaging contrast agents, photothermal agents, photoacoustic agents, and radiation dose enhancers, among other applications. Recent advances in nanotechnology have led to the use of nanomaterials
in many areas of functional imaging, cancer therapy.\\ Nanoparticles 1 – 100 nm in diameter have dimensions comparable to biological functional units. Diverse surface chemistries, unique magnetic properties, tunable absorption and emission properties, and recent advances in the synthesis and engineering of various nanoparticles suggest their potential as probes for early detection of diseases such as cancer.
 


\section*{Week 6 }
\subsection*{ Quantum dots in biomedical imaging}
Quantum specks are progressively utilized as fluorophores for in vivo fluorescence imaging. Fluorescence imaging enjoys a few benefits contrasted and other imaging modalities since this strategy has great responsiveness and is painless in nature, utilizing promptly accessible and generally economical instruments. Being an optical procedure, it is restricted as far as tissue infiltration profundity. A wide assortment of in vivo examinations have approved the strength of QDs. Akerman et al. have exhibited critical collection of CdSe/Zns QDs covered with PEG and a lung-focusing on peptide in the lungs of mice .\\Gao et al. as of late portrayed the improvement of multifunctional nanoparticle tests in light of QDs for in vivo imaging of human prostate disease in mice. This new class of covered QDs depends on the exemplification of PEGylated QDs utilizing an ABC triblock copolymer as an optional covering layer, further functionalized with a cancer focusing on neutralizer to prostate-explicit film antigen [44]. Cai et al utilized peptide-formed QDs for harmless, focused on in vivo imaging of cancers. They showed that QDs named with arginine-glycine-aspartic corrosive (RGD) peptide specifically focus on the αvβ3-positive cancer vasculature in a murine xenograft model, as seen from close to infrared (NIR) fluorescence pictures [67]. Despite the fact that there was huge amassing in the liver, bone marrow and lymph hubs, 6 h after the infusion of QDs, high growth contrast was additionally noticed . Ballou et al. concentrated on the limitation of four QDs with various surface coatings and exhibited that the QDs stayed fluorescent for somewhere around 4 months in vivo and that the confinement of QDs was directed by the surface covering
\begin{itemize}
     
    
\end{itemize}





 
 \end{document}
