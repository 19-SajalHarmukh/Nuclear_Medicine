\documentclass{article}
\usepackage[paper=a4paper,left=3cm, right=3cm, top=2cm]{geometry}
\usepackage[utf8]{inputenc}

\title{\textbf{Nanoparticles : Medical Imaging Advancement\\Update 3}}
\author{Sajal Harmukh \\ \textit{Department of Biomedical Engineering} \\ National Institute of Technology, Raipur }
\date{04\textsuperscript{th} February 2022}

\begin{document}

\maketitle

\section*{Aim}
\large
To study and summarise the recent advancement and the emerging research on the use of nanoparticles in the biomedical imaging and to lookout for an early detection and diagnosis of the disease, which is the driving force in the field of medical imaging.

\section*{Introduction}
Nanomaterials, such as nanoparticles, nanorods, nanosphere, nanoshells, and nanostars, are
very commonly used in biomedical imaging.They make excellent drug carriers, imaging contrast agents, photothermal agents, photoacoustic agents, and radiation dose enhancers, among other applications. Recent advances in nanotechnology have led to the use of nanomaterials
in many areas of functional imaging, cancer therapy.\\ Nanoparticles 1 – 100 nm in diameter have dimensions comparable to biological functional units. Diverse surface chemistries, unique magnetic properties, tunable absorption and emission properties, and recent advances in the synthesis and engineering of various nanoparticles suggest their potential as probes for early detection of diseases such as cancer.
 


\section*{Week 3}
Not withstanding inactive focusing on procedures, nanoparticle surface naming with different ligands to target receptors can build imaging contrast specialist restriction through explicit restricting to target receptors in injuries. For instance, gold nanoparticles surface embellished with a prostate-explicit layer antigen RNA aptamer have been shown a higher CT thickness for prostate disease cell imaging. What's more, nano-sized superparamagnetic iron oxide (SPIO) specialists surface adorned with a high-liking against EGFR immune response have been displayed to target lung growths by MRI.\\Antibodies and counter acting agent pieces are the most well-known and productive dynamic focusing on ligands. Antibodies have a high explicit partiality to the comparing antigens which can increment nanoparticle fixation to a particular location.20 Another ligand utilized for focusing on is an aptamer, which is additionally named as a synthetic counter acting agent. It is a solitary DNA or RNA grouping that folds into an optional design with a high focusing on fondness to cell surface receptors. Contrasted and antibodies, aptamers are little, simple to blend and give lower immunogenicity.\\ In addition to active and passive targeting strategies, various stimuli also play a targeting role in nanoparticle imaging applications, In these physical targeting strategies, external sources or fields guide nanoparticles to the target site and control the release process, as seen in photothermal and magnetic hyperthermia therapy. An acidic pH/reduction dual-stimuli responsive nanoprobe for enhanced CT imaging of tumor is another example.


 
  

\end{document}
