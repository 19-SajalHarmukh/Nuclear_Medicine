\documentclass{article}
\usepackage[paper=a4paper,left=3cm, right=3cm, top=2cm]{geometry}
\usepackage[utf8]{inputenc}

\title{\textbf{Nanoparticles : Medical Imaging Advancement\\Update 11}}
\author{Sajal Harmukh \\ \textit{Department of Biomedical Engineering} \\ National Institute of Technology, Raipur }
\date{01\textsuperscript{st} April 2022}

\begin{document}

\maketitle

\section*{Aim}
\large
To study and summarise the recent advancement and the emerging research on the use of nanoparticles in the biomedical imaging and to lookout for an early detection and diagnosis of the disease, which is the driving force in the field of medical imaging.

\section*{Introduction}
Nanomaterials, such as nanoparticles, nanorods, nanosphere, nanoshells, and nanostars, are
very commonly used in biomedical imaging.They make excellent drug carriers, imaging contrast agents, photothermal agents, photoacoustic agents, and radiation dose enhancers, among other applications. Recent advances in nanotechnology have led to the use of nanomaterials
in many areas of functional imaging, cancer therapy.\\ Nanoparticles 1 – 100 nm in diameter have dimensions comparable to biological functional units. Diverse surface chemistries, unique magnetic properties, tunable absorption and emission properties, and recent advances in the synthesis and engineering of various nanoparticles suggest their potential as probes for early detection of diseases such as cancer.
 


\section*{Week 11 }
\subsection*{ Nanotechnology in drug delivery }
Treatment normally includes conveying medications to a particular objective site. In the event that an interior course for drug conveyance isn't accessible, outer restorative strategies, for example, radiotherapy and surgeries are utilized. These techniques are regularly utilized conversely or in mix to battle infections. The objective of treatment is to constantly specifically eliminate the cancers or the wellspring of disease in a durable way . Nanotechnologies are making a convincing commitment around here through the improvement of novel modes for drug conveyance, and a portion of these strategies have demonstrated successful in a clinical setting and are clinically utilized . For instance, doxorubicin a medication which displays high poisonousness, can be conveyed straightforwardly to cancer cells utilizing liposomes  without influencing the heart or kidneys. Moreover, paclitaxel fused with polymeric mPEG-PLA micelles are utilized in chemotherapeutic therapy of metastatic bosom malignant growths . The progress of nanotechnologies in drug conveyance can be ascribed to the worked on in vivo dissemination, avoidance of the reticuloendothelial framework and the positive pharmacokinetics .

 \end{document}
