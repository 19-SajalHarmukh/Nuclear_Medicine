\documentclass{article}
\usepackage[paper=a4paper,left=3cm, right=3cm, top=2cm]{geometry}
\usepackage[utf8]{inputenc}

\title{\textbf{Nanoparticles : Medical Imaging Advancement\\Update 7}}
\author{Sajal Harmukh \\ \textit{Department of Biomedical Engineering} \\ National Institute of Technology, Raipur }
\date{04\textsuperscript{th} March 2022}

\begin{document}

\maketitle

\section*{Aim}
\large
To study and summarise the recent advancement and the emerging research on the use of nanoparticles in the biomedical imaging and to lookout for an early detection and diagnosis of the disease, which is the driving force in the field of medical imaging.

\section*{Introduction}
Nanomaterials, such as nanoparticles, nanorods, nanosphere, nanoshells, and nanostars, are
very commonly used in biomedical imaging.They make excellent drug carriers, imaging contrast agents, photothermal agents, photoacoustic agents, and radiation dose enhancers, among other applications. Recent advances in nanotechnology have led to the use of nanomaterials
in many areas of functional imaging, cancer therapy.\\ Nanoparticles 1 – 100 nm in diameter have dimensions comparable to biological functional units. Diverse surface chemistries, unique magnetic properties, tunable absorption and emission properties, and recent advances in the synthesis and engineering of various nanoparticles suggest their potential as probes for early detection of diseases such as cancer.
 


\section*{Week 7 }
\subsection*{ Surface modification and bioconjugation}
The need of hydrophobic circumstances for combination of excellent QDs hinders direct exchange to natural applications, as numerous biomolecules have restricted dissolvability and dependability in natural solvents . Like gold nanoparticles, ligand trade or formation to surface stabilizers can be utilized for surface change of QDs to further develop solidness in watery circumstances and biocompatibility . In the ligand trade process, heterobifunctional ligands, for example, mercaptoacetic corrosive or 3-mercaptopropyl trimethoxy silane containing thiol functionalities are utilized, which can covalently tie to QDs. The corrosive functionalities further develop QD hydrophilicity . Notwithstanding, the ligand trade technique might initiate agglomeration and lessening fluorescence productivity. One more way to deal with work on the steadiness and dissolvability in watery circumstances is surface adjustment utilizing amphiphilic polymers . Altered QDs with expanded water solidness may then be formed with explicit ligands, for example, peptides, antibodies or little particles to give target explicitness . Like gold nanoparticles, PEG can be formed to QDs possibly to expand the blood dissemination time and lessen vague restricting to serum proteins in blood
\begin{itemize}
     
    
\end{itemize}





 
 \end{document}
