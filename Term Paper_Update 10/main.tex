\documentclass{article}
\usepackage[paper=a4paper,left=3cm, right=3cm, top=2cm]{geometry}
\usepackage[utf8]{inputenc}

\title{\textbf{Nanoparticles : Medical Imaging Advancement\\Update 10}}
\author{Sajal Harmukh \\ \textit{Department of Biomedical Engineering} \\ National Institute of Technology, Raipur }
\date{25\textsuperscript{th} March 2022}

\begin{document}

\maketitle

\section*{Aim}
\large
To study and summarise the recent advancement and the emerging research on the use of nanoparticles in the biomedical imaging and to lookout for an early detection and diagnosis of the disease, which is the driving force in the field of medical imaging.

\section*{Introduction}
Nanomaterials, such as nanoparticles, nanorods, nanosphere, nanoshells, and nanostars, are
very commonly used in biomedical imaging.They make excellent drug carriers, imaging contrast agents, photothermal agents, photoacoustic agents, and radiation dose enhancers, among other applications. Recent advances in nanotechnology have led to the use of nanomaterials
in many areas of functional imaging, cancer therapy.\\ Nanoparticles 1 – 100 nm in diameter have dimensions comparable to biological functional units. Diverse surface chemistries, unique magnetic properties, tunable absorption and emission properties, and recent advances in the synthesis and engineering of various nanoparticles suggest their potential as probes for early detection of diseases such as cancer.
 


\section*{Week 10 }
\subsection*{ Dendrimers }
Dendrimers are macromolecules with branched repeating units expanding from a central core and consists of exterior functional groups . These functional groups can be anionic, neutral or cationic terminals, and they can be used to modify the entire structure, and/or the chemical and physical properties. Therapeutic agents can be encapsulated within the interior space of dendrimers, or attached to the surface groups, making dendrimers highly bioavailable and biodegradable. Conjugates of dendrimers with saccharides or peptides have been shown to exhibit enhanced antimicrobial, antiprion and antiviral properties with improved solubility and stability upon absorption of therapeutic drugs. Polyamidoamine dendrimer-DNA complexes (called dendriplexes) have been investigated as gene delivery vectors and hold promise in facilitating successive gene expression, targeted drug delivery and improve drug efficacy . dendrimers are promising particulate systems for biomedical applications, such as in imaging and drug delivery , due to their transformable properties.

 \end{document}
