\documentclass{article}
\usepackage[paper=a4paper,left=3cm, right=3cm, top=2cm]{geometry}
\usepackage[utf8]{inputenc}

\title{\textbf{Nanoparticles : Medical Imaging Advancement\\Update 9}}
\author{Sajal Harmukh \\ \textit{Department of Biomedical Engineering} \\ National Institute of Technology, Raipur }
\date{18\textsuperscript{th} March 2022}

\begin{document}

\maketitle

\section*{Aim}
\large
To study and summarise the recent advancement and the emerging research on the use of nanoparticles in the biomedical imaging and to lookout for an early detection and diagnosis of the disease, which is the driving force in the field of medical imaging.

\section*{Introduction}
Nanomaterials, such as nanoparticles, nanorods, nanosphere, nanoshells, and nanostars, are
very commonly used in biomedical imaging.They make excellent drug carriers, imaging contrast agents, photothermal agents, photoacoustic agents, and radiation dose enhancers, among other applications. Recent advances in nanotechnology have led to the use of nanomaterials
in many areas of functional imaging, cancer therapy.\\ Nanoparticles 1 – 100 nm in diameter have dimensions comparable to biological functional units. Diverse surface chemistries, unique magnetic properties, tunable absorption and emission properties, and recent advances in the synthesis and engineering of various nanoparticles suggest their potential as probes for early detection of diseases such as cancer.
 


\section*{Week 9 }
\subsection*{ Iron oxide nanoparticles}
Attractive nanoparticles stand out due to their likely use in optical, attractive and electronic gadgets . Like gold nanoparticles and QDs, metal nanoparticles of iron oxide are relied upon to show OK biocompatibility at low focus, high attractive immersion and useful surfaces. Attractive iron oxide nanoparticles have been functionalized with antibodies, nucleosides, proteins and chemicals for guiding them to sick tissues, for example, growths . Iron oxide nanoparticles show superparamagnetism and high field irreversibility, which emerge partially from the size and surface properties of individual nanoparticles . For over twenty years, attractive nanoparticles have been utilized effectively clinically. The compound and underlying model of attractive nanoparticles is like gold nanoparticles and QDs. Profoundly superparamagnetic iron oxide is usually utilized as the center material, and biocompatible polymers, for example, dextran might be utilized as a covering material . Superparamagnetic iron oxide nanoparticles (SPION) have huge attractive minutes and are appropriate as T2 contrast specialists in MRI. In vivo, SPIONs might be coordinated to the reticuloendothelial framework (RES) not set in stone by the molecule size and surface science. SPIONs are additionally grouped into crosslinked iron oxide nanoparticles (CLIO) , huge SPIO (iron oxide nanoparticles with polymer covering materials) and tiny SPIO .
\begin{itemize}
     
    
\end{itemize}





 
 \end{document}
